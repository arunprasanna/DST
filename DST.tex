%%%%%%%%%%%%%%%%%%%%%%%%%%%%%%%%%%%%%%%%%
% Short Sectioned Assignment
% LaTeX Template
% Version 1.0 (5/5/12)
%
% This template has been downloaded from:
% http://www.LaTeXTemplates.com
%
% Original author:
% Frits Wenneker (http://www.howtotex.com)
%
% License:
% CC BY-NC-SA 3.0 (http://creativecommons.org/licenses/by-nc-sa/3.0/)
%
%%%%%%%%%%%%%%%%%%%%%%%%%%%%%%%%%%%%%%%%%

%----------------------------------------------------------------------------------------
%	PACKAGES AND OTHER DOCUMENT CONFIGURATIONS
%----------------------------------------------------------------------------------------
\documentclass[paper=a4, fontsize=12pt]{scrartcl} % A4 paper and 11pt font size

\usepackage{hyperref}

\usepackage{graphicx} % Required for the inclusion of images
\usepackage{xcolor}

\usepackage[T1]{fontenc} % Use 8-bit encoding that has 256 glyphs
\usepackage{fourier} % Use the Adobe Utopia font for the document - comment this line to return to the LaTeX default
\usepackage[english]{babel} % English language/hyphenation
\usepackage{amsmath,amsfonts,amsthm} % Math packages
\usepackage{listings} % Required for insertion of code
\lstset { %
    language=C++,
    backgroundcolor=\color{black!5}, % set backgroundcolor
    basicstyle=\footnotesize,% basic font setting
}

\usepackage{lipsum} % Used for inserting dummy 'Lorem ipsum' text into the template

\usepackage{sectsty} % Allows customizing section commands
\allsectionsfont{\centering \normalfont\scshape} % Make all sections centered, the default font and small caps
 


\usepackage{fancyhdr} % Custom headers and footers
\pagestyle{fancyplain} % Makes all pages in the document conform to the custom headers and footers
\fancyhead{} % No page header - if you want one, create it in the same way as the footers below
\fancyfoot[L]{} % Empty left footer
\fancyfoot[C]{} % Empty center footer
\fancyfoot[R]{\thepage} % Page numbering for right footer
\renewcommand{\headrulewidth}{0pt} % Remove header underlines
\renewcommand{\footrulewidth}{0pt} % Remove footer underlines
\setlength{\headheight}{13.6pt} % Customize the height of the header

\numberwithin{equation}{section} % Number equations within sections (i.e. 1.1, 1.2, 2.1, 2.2 instead of 1, 2, 3, 4)
\numberwithin{figure}{section} % Number figures within sections (i.e. 1.1, 1.2, 2.1, 2.2 instead of 1, 2, 3, 4)
\numberwithin{table}{section} % Number tables within sections (i.e. 1.1, 1.2, 2.1, 2.2 instead of 1, 2, 3, 4)

\setlength\parindent{0pt} % Removes all indentation from paragraphs - comment this line for an assignment with lots of text

%----------------------------------------------------------------------------------------
%	TITLE SECTION
%----------------------------------------------------------------------------------------

\newcommand{\horrule}[1]{\rule{\linewidth}{#1}} % Create horizontal rule command with 1 argument of height

\title{	
\normalfont \normalsize 
\textsc{Georgia Institute of Technology, ECE} \\ [25pt] % Your university, school and/or department name(s)
\horrule{0.5pt} \\[0.4cm] % Thin top horizontal rule
\huge Logic Simulator \\ % The assignment title
\horrule{2pt} \\[0.5cm] % Thick bottom horizontal rule
}


\author{} % Your name



\date{\normalsize\today} % Today's date or a custom date

\begin{document}

\maketitle 

% Print the title
%----------------------------------------------------------------------------------------
%	PROBLEM 1
%----------------------------------------------------------------------------------------
%\center {GTID: 903062365\newline}
\begin{flushright}
Arunprasanna Sundararjan Poorna\newline
\end{flushright}

\newpage
\tableofcontents
\newpage
\vspace{60pt}
\section{Introduction}
The objective of this project is to simulate logic gates efficiently. All gates have been pre processed and reduced to two input gates, thus making it easy to parse and simulate each gate. We maintain a stack of gates that can be evaluated and for each gate that is evaluated we check if the output of the gate translates to any of the inputs to other gates. This report describes the data structures used and provides the results for the inputs given to the simulator. 
%------------------------------------------------
\newpage
\section{Data Structures Used}
This section describes the data structures used to implement the logic simulator. The project has been done in Python 2.7.X, and I have described the data structures in detailed below:
\begin{itemize}
\item There are several parameters stored as integer variables and these are: number of gates, number of nets, input nets, output nets etc. 
\item A class has been defined to wrap the variables required to define a gate and it's methods. This class is called \emph{gates} 
	\begin {itemize}
	\item The class has a string "gate" that stores what type of gate it is. It also has integer variables to denote the inputs and the outputs.  
	\item The class also has several methods. It has an overloaded constructor that can be called to create an object as and when the input file is being read. 
	\item The class also has methods to display the information about the gate and evaluate the gate itself. 
	\end {itemize}
\item A Dictionary in Python is a resizable hash table that stores key,value pairs. It has O(1) lookup for elements. 
\item A dictionary has been used to store the gate number, gate objects pairs
\item A dictionary has been used to store the net number, net value pairs.
\item A dictionary has been used to map each net to a list of gates that it drives. So that this can be looked up to find which gates may have both inputs to be valid after evaluating a gate. 
	\begin {itemize}
	\item This dictionary is imported from collections module since the native dictionary is insufficient.
	\item We need a special dictionary that can help us store a linked list of values for each key. 
	\end{itemize}
\item Note that the dictionary(hash table) is a powerful data structure, and it's power lies in the flexibility of storing key, value pairs which can be of any type.
\item A list has been used as a stack in Python by using the .append() and .pop() methods. The stack contains a stack of objects. 
\end{itemize} 
	
\newpage
\section{Pseudo Code}
This section describes the pseudo code of the algorithm used for the logic simulator. 
\begin{itemize}
\item Do a single pass of the file, to find the number of gates, the maximum number of nets and also store the input and output gates. 
\item Initialize the dictionaries used(refer above), with default  constructors to store -1 in inputs and outputs to indicate that they are invalid entries. 
\item  Read the file line by line, creating objects of type gate and storing them in a dictionary with the gate no. (line no.) as the key. 
\item The type of gate is read and the corresponding constructor is invoked to create the object. Input2 is not valid for gates that are INV or BUF. 
\item A bit vector of required input length(number of inputs) is taken as input. 
\item Display an error if the length of input is incorrect and re-prompt for correct input. 
\item The corresponding net values are assigned based on the bit vector input. 
\item All these bits are converted to boolean, since Python uses Boolean logic for all bit wise operators. This is done by the function bit\_to\_boolean(bit). 
\item A corollary of the above function is also defined to convert the final boolean output to bits. This function is called boolean\_to\_bit(boolean).
\item Go through all the gate objects and push those gates that have valid bits in all their inputs(not -1) into a stack.  
\item While the stack is not empty. 
\begin{itemize}
	\item Pop the gate object.
	\item Display the contents of the object. 
	\item Evaluate the gate. 
	\item Go through all the values in the list of values contained in the dictionary that maps nets to the gates the are given as input to. 
	\item Go through the above value which corresponds to a gate and see if both the inputs of the gate are now valid. 
	\item Push it into the stack if the inputs are now valid. 
\end{itemize}
\item Convert the boolean output to bit output using the function boolean\_to\_bit(boolean)
\item Print the final bit vector obtained. 
\end{itemize}

\newpage
\section{Results}
\subsection{s27.txt}
\begin{center}
\begin{tabular}{| c | c | }
\hline INPUT VECTORS &	OUTPUT VECTORS \\ \hline
1110101 & 1001 \\ \hline
0001010 &  0100 \\ \hline
1010101 & 1001 \\ \hline
0110111 & 0001 \\ \hline
1010001 & 1001 \\ \hline

\end{tabular}
\end{center}
\subsection{s298f\_2.txt}
\begin{center}
\begin{tabular}{| c | c | }
\hline INPUT VECTORS & OUTPUT VECTORS \\ \hline
10101010101010101 & 00000010101000111000 \\ \hline
01011110000000111 & 00000000011000001000 \\ \hline
11111000001111000	& 00000000001111010010 \\ \hline
11100001110001100 & 00000000100100100101 \\ \hline
01111011110000000 & 11111011110000101101 \\ \hline

\end{tabular}
\end{center}
\subsection{s344f\_2.txt}
\begin{center}
\begin{tabular}{| c | c | }
\hline INPUT VECTORS & OUTPUT VECTORS \\ \hline
101010101010101011111111 & 10101010101010101010101101 \\ \hline
010111100000001110000000 & 00011110000000100001111100 \\ \hline
111110000011110001111111 & 00011100000111011000111010 \\ \hline
111000011100011000000000 & 00001101111001111111000010 \\ \hline
011110111100000001111111 & 10011101111000001001000100 \\ \hline

\end{tabular}
\end{center}
\subsection{s349f\_2.txt}
\begin{center}
\begin{tabular}{| c | c | }
\hline INPUT VECTORS & OUTPUT VECTORS \\ \hline
101010101010101011111111 & 10101010101010101101010101 \\ \hline
010111100000001110000000 & 00011110000000101011110000 \\  \hline
111110000011110001111111 & 00011100000111010001111100 \\ \hline
111000011100011000000000 & 00001101111001110010001111 \\ \hline
011110111100000001111111 & 10011101111000001010000100 \\ \hline

\end{tabular}
\end{center}
\newpage
\section{Appendix: Code}
\lstset {language=Python}
\begin{lstlisting}
import os 
import glob 
import sys 
from collections import defaultdict

files = []
for file in glob.glob("*.txt"):
	files.append(file)

print files
def bit_to_boolean(bit):
	if "1" in bit:
		return True
	else:
		return False

def boolean_to_bit(boolean):
	if boolean:
		return "1"
	else:
		return "0"

class gates:
	def __init__(self):
		self.gate =-1
		self.input1 = -1
		self.input2 = -1
		self.output = -1
	def __init__(self, string, input1, output, input2=None):
		if(input2==None):
			self.gate = string
			self.input1 = input1
			self.input2 = -1
			self.output = output
		else:
			self.gate = string
			self.input1 = input1
			self.input2 = output
			self.output = input2

	def display(self):
		print "Gate is: " + self.gate
		print "Inputs are",
		print self.input1, self.input2
		print "Output is:",
		print self.output
	def evaluate(self, nets):
		if "AND" in self.gate:
			nets[self.output] = nets[self.input1] and nets[self.input2]
		if "OR" in self.gate:
			nets[self.output] = nets[self.input1] or nets[self.input2]
		if "NAND" in self.gate:
			nets[self.output] = not(nets[self.input1] and nets[self.input2])
		if "NOR" in self.gate:
			nets[self.output] = not(nets[self.input1] or nets[self.input2])
		if "INV" in self.gate:
			nets[self.output] = not(nets[self.input1])
		if "BUF" in self.gate:
			nets[self.output] = nets[self.input1]

for file_ in files[1:2]:
	lines = []
	print "\n\n"
	print file_
	print "\n\n"
	with open(file_) as inputfile:
		for line in inputfile:
			lines.append(line.split("\r\n")[0])
	lines = lines[:len(lines)]
	number_of_gates =0
	number_of_nets = 0
	inputs =[]
	outputs =[]
	both_input_list = ["AND", "OR", "NAND", "NOR"]
	single_input_list = ["INV", "BUF"]
	for index, line in enumerate(lines):
		if "INPUT" in str(line):
			number_of_gates = index
	for line in lines:
		if "INPUT" in str(line):
			inputs= (str(line).split()[1:len(str(line).split())-1])
	for line in lines:
		if "OUTPUT" in str(line):
			outputs= (str(line).split()[1:len(str(line).split())-1])
	max_lines = map(int,lines[number_of_gates-1].split()[1:])
	number_of_nets= max(max_lines)
	nets ={}
	nets_inputgates =defaultdict(list)
	for i in range(1, number_of_nets+1):
		nets[i] = -1
	print "Gates # :",
	print number_of_gates
	dict_of_gates ={}
	for i in range(number_of_gates):
		if len(lines[i].split()) == 3:
			a = gates(lines[i].split()[0], int(lines[i].split()[1]), int(lines[i].split()[2]))
			dict_of_gates[i+1]=a
			nets_inputgates[int(lines[i].split()[1])].append(i+1)
		if len(lines[i].split()) ==4:
			b = gates(lines[i].split()[0], int(lines[i].split()[1]), int(lines[i].split()[2]), int(lines[i].split()[3]))
			dict_of_gates[i+1]=b
			nets_inputgates[int(lines[i].split()[1])].append(i+1)
			nets_inputgates[int(lines[i].split()[2])].append(i+1)
	inputs = map(int, inputs)
	outputs = map(int, outputs)
	print inputs
	print outputs
	print "Please provide the bit vector of length:",
	print len(inputs)
	bit_vector = raw_input()
	if len(bit_vector) != len(inputs):
		print "Please enter a bit vector of length:",
		print len(inputs)
		bit_vector = raw_input()
	print bit_vector
	input_vector = []
	for char in bit_vector:
		input_vector.append(bit_to_boolean(char))
	print input_vector
	stack = []
	for input_, char in zip(inputs, input_vector):
		nets[input_] = char
	for gate_number, gate_object in dict_of_gates.iteritems():
		if gate_object.gate in both_input_list:
			if nets[gate_object.input1] !=-1 and nets[gate_object.input2] != -1 :
				stack.append(gate_object)
		if gate_object.gate in single_input_list:
			if nets[gate_object.input1] != -1:
				stack.append(gate_object)

	while stack:
		gate_object = stack.pop()
		gate_object.display()
		gate_object.evaluate(nets)
		for values in nets_inputgates[gate_object.output]:
			check_object= dict_of_gates[values]
			if check_object.gate in both_input_list:
				if nets[check_object.input1] != -1 and nets[check_object.input2] != -1:
					stack.append(check_object)
			if check_object.gate in single_input_list:
				if nets[check_object.input1] != -1:
					stack.append(check_object)

	final_bit = ""

	for output in outputs:
		print output,
		print nets[output]
		final_bit = final_bit+ boolean_to_bit(nets[output])
	
	print final_bit

\end{lstlisting}


%----------------------------------------------------------------------------------------

\end{document}